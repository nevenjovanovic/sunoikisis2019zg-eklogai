%-*-coding:utf-8;-*-
\documentclass{beamer}
\usepackage{polyglossia}
\setdefaultlanguage{english}
\setotherlanguage[variant=ancient]{greek}

\usepackage{fontspec}
\usepackage{verse}
\usepackage{tabto}
\defaultfontfeatures{Ligatures=TeX}
%\usepackage[usenames,dvipsnames,svgnames,table]{xcolor}
\definecolor{links}{HTML}{2A1B81}
\hypersetup{colorlinks,linkcolor=,urlcolor=links}
\usetheme[numbering=none]{metropolis}           % Use metropolis theme
\setsansfont{Kyrtos}
\hyphenation{me-mo-ra-bi-li-um}
\title{From Annotated Text to Vocabulary Exercises}
\subtitle{Sunoikisis Digital Classics, Summer 2019}
\date{}
\author{Thursday June 13, 17:00 - 18:15 CEST\\Neven Jovanović, Petar Soldo}
\institute{University of Zagreb\\Faculty of Humanities and Social Sciences\\Department of Classical Philology \\
\href{https://github.com/SunoikisisDC/SunoikisisDC-2018-2019/wiki/Summer2019-Session11}{github.com/SunoikisisDC/SunoikisisDC-2018-2019/wiki/Summer2019-Session11}}
\begin{document}
  \maketitle



\section*{The plan}

\begin{frame}{The plan}

\tableofcontents


\end{frame}

\section{The task}

\begin{frame}{The task}

Use the set of annotated Greek texts to learn its vocabulary.

\end{frame}

\section{Materials and methods}

\begin{frame}{Linguistic annotation}

Lemmatize words in a text

Add morphological descriptions of lemmatized words (parts of speech)

\end{frame}

\begin{frame}{XML manipulation}

\alert{Reorder} words by form, lemma, part of speech

\alert{Regroup} words by forms, lemmata, parts of speech (a group is a set of identical forms, identical lemmata etc; from it we get information about frequency of an item in the collection)

Analyze frequencies, retrieve reoccurring forms, lemmata, parts of speech

Select certain forms, lemmata, parts of speech

\end{frame}

\begin{frame}{XML manipulation, common tasks}

Retrieve all forms from a text (but not punctuation)

For a form, retrieve lemma: πράξεις $\rightarrow$ πρᾶξις

For a form, retrieve part of speech information: \\
πράξεις $\rightarrow$ \texttt{n - p - - - fa -}


\end{frame}

\begin{frame}{XML manipulation, common tasks}

For a lemma, find whether it occurs in other texts, or in another database (DC Greek Core)

For a lemma, retrieve form in a text

For a lemma, retrieve (Croatian) meaning from a lexical database (DC Greek Core)


\end{frame}

\begin{frame}{XML manipulation, common tasks}

Format results for import into Anki, our vocabulary learning program (a simple three-fields CSV is needed)

\end{frame}

\begin{frame}{Vocabulary learning}

\alert{Spaced repetition}: we tend to remember things more effectively if we spread reviews out over time, instead of studying multiple times in one session.

\end{frame}

\begin{frame}{Vocabulary learning}

\alert{Make your own cards!}

The act of making the card helps you remember the stuff on it.

The way you learn and the things you care about are unique to you.

\end{frame}

\section{Tools}

\begin{frame}{Linguistic annotation}

Arethusa / Perseids

\end{frame}

\begin{frame}{XML manipulation}

BaseX / XQuery

\end{frame}

\begin{frame}{Spaced repetition}

Anki

\end{frame}

\begin{frame}{Frequent words in (classical) Greek and Latin}

\alert{Dickinson College Core Vocabulary}

``The DCC Core Vocabulary lists represent the thousand most common words in Latin and the 500 most common words in ancient Greek. They were originally composed in 2012–13 by a team at Dickinson College led by \alert{Christopher Francese}. The lists are downloadable in various formats and licensed under a Creative Commons Attribution-ShareAlike 3.0 Unported License.''

\end{frame}

\section{Creating the database}

\begin{frame}{F1}


\end{frame}

\section{Creating the exercises}

\begin{frame}{Exercise scenarios}

1. For a word form, give the lemma (standard vocabulary entry):\\
Q πράξεις, A πρᾶξις

\end{frame}

\begin{frame}{Exercise scenarios}


2. For a word form, give the grammatical description:\\
Q πράξεις, A noun, feminine, plural, nominative / accusative / vocative

\end{frame}

\begin{frame}{Exercise scenarios}

3. For a lemma (of a word from the DC Greek core list), give the translation:\\
Q ποταμός, A rijeka

\end{frame}

\section{Demonstration (results)}

\begin{frame}{F1}


\end{frame}

\section{Discussion}

\begin{frame}{F1}


\end{frame}



  \maketitle


\end{document}
{
    \usebackgroundtemplate{\includegraphics[height=\paperheight]{img/housman.jpg}}
    \setbeamertemplate{navigation symbols}{}
    \begin{frame}[plain]
    \end{frame}
    }




{
    \usebackgroundtemplate{\parbox[c][\paperheight][c]{\paperwidth}{\centering\includegraphics[width=\paperwidth]{img/platocritilatl12.png}}}
    \setbeamertemplate{navigation symbols}{}
    \begin{frame}[plain]
    \end{frame}
    }

